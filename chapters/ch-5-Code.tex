\chapter{Codice}

\section{Server}

\subsection{Routes}
La definizione delle Route è situata nei file della cartella \textit{routes}.\newline
Il router del server crea la route attraverso un metodo che ha lo stesso nome del metodo HTTP usato per accedere alla route e vi associa l'handler che deve gestire le request.\newline
È possibile associare anche dei middleware alle route, ovvero una funzione che viene eseguita prima dell'effettivo handler, solitamente per effettuare dei controlli.\newline
Nello snippet che segue il middleware auth non permette lo scaricamento di un report, a meno che l'utente non sia autenticato.\newline
Tutti i componenti che si occupano di gestire la logica delle route si trovano nella cartella \textit{controller}.

\begin{minted}{js}
router.post("/auth/signup",(req, res) => AuthController.signup(req, res));

router.post("/auth/login", (req, res) => AuthController.login(req, res));
//notice the auth middleware
router.get("/dw/report", auth, (req, res) => ResourceController.downloadReport(req, res));

\end{minted}

\subsection{Model su Mongoose}
La definizione degli Schema del DataBase sono state strutturate utilizzando Mongoose. Lo Schema degli User per esempio è definito nel seguente modo:

\begin{minted}{js}
let UserSchema = new Schema({
    username: {
        type: String,
        require: true
    },
    password: {
        type: String,
        require: true
    },
    notifications: [{
        type: Schema.Types.ObjectId,
        ref:"Notification"
    }],
    user_in_games: [{
        type: Schema.Types.ObjectId,
        ref:"UserInGame"
    }]
});
\end{minted}

\noindent Anche tutti gli altri Schema sono stati creati utilizzando la stessa sintassi. È possibile salvare un nuovo utente creandolo attraverso il costruttore e poi richiamando il metodo \textit{save()}.
\begin{minted}{js}
 let newUser = new User({username: username, password: pw});
 newUser.save().then(
                //handler if operation succeeds
            ).catch(
                //handler if operation fails
            );
\end{minted}

\noindent Per effettuare operazioni direttamente sulla collezione è possibile utilizzare direttamente i metodi offerti dallo schema:

\begin{minted}{js}
 User.findOne({ username: "Jotaro" }).then(result => /*...*/)
\end{minted}
\subsection{Socket Handlers}
Una volta che una connessione tramite socket viene instaurata col client è possibile ricevere ed inviare messaggi attraverso un canale full duplex. Ogni messaggio viene mandato e ricevuto su uno specifico channel, in modo che un handler possa leggerlo, gestirlo, e con ogni probabilità aggiornare lo stato interno del sistema.
Gli handler dei canali delle socket si trovano nella cartella \textit{socket/handlers}, mentre il controller che li utilizza è il file \textit{socket/controller.js}.
\begin{minted}{js}
    io.on('connection', (socket) => {

        socket.on(Channels.CREATE_LOBBY, (json) => createLobbyHandler(socket, json));
    
        socket.on(Channels.SENTENCE, (json) => sentenceHandler(socket, json));

        socket.on(Channels.DRAW, (json) => drawHandler(socket, json));
        
        //...
        
        socket.on(Channels.DISCONNECT, () => disconnectHandler(socket));
    });
\end{minted}

\section{Client}

\subsection{Requests}
La gestione delle request sul client è incapsulata nei file js il cui nome finisce con \textit{"Logic"}, i quali mettono a disposizione funzioni che effettuano richieste e aggiornano lo stato interno dell'applicazione.\newline
Per effettuare le richieste è stata utilizzata la libreria JQuery. Una richiesta ha il seguente formato: 
\begin{minted}{js}
    ajax({
    	url: SERVER_ADDRESS + "/route",
    	type: 'POST', //can be for example GET, or any other HTTP method
    	data: {...} //data to send, not mandatory
    	headers: {"Authorization": token} //used in protected routes
    }).done(function (result) {
    	// do something with the result... probably updating redux status
    });
\end{minted}

L'header \textit{"Authorizarion"} è necessario solo per le route che richiedono che l'utente sia autenticato. In questo caso il valore è il Bearer token affidato all'utente, incapsulato nello stato di redux.

\subsection{Socket Handlers}
Una volta che una connessione tramite socket viene instaurata col server è possibile ricevere ed inviare messaggi attraverso un canale full duplex. Ogni messaggio viene mandato e ricevuto su uno specifico channel, in modo che un handler possa leggerlo, gestirlo, e con ogni probabilità aggiornare lo stato interno del sistema.
Gli handler dei canali delle socket si trovano nella cartella \textit{socket/handlers}
\begin{minted}{js}
    io.on('connection', (socket) => {

        socket.on(Channels.CREATE_LOBBY, (json) => createLobbyHandler(socket, json));
    
        socket.on(Channels.SENTENCE, (json) => sentenceHandler(socket, json));

        socket.on(Channels.DRAW, (json) => drawHandler(socket, json));
        
        [...]
        
        socket.on(Channels.DISCONNECT, () => disconnectHandler(socket));
    });
\end{minted}

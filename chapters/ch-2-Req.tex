\chapter{Requisiti}
L’idea di progetto consiste nel creare un’architettura Client-Server che permetta lo svolgimento del gioco sopra descritto tramite un client web.\newline
\noindent Il server coordina i giocatori, orchestrandone gli stati, con l’idea di rendere il gioco scalabile.\newline

\noindent Si vuole rendere fruibile il gioco tramite client web mobile, tablet e desktop, per permettere a più giocatori possibile di giocare garantendo totale libertà riguardo al device scelto; ne segue dunque che il client sarà realizzato seguendo una filosofia mobile first.\newline

\noindent Sarà possibile registrarsi al servizio, anche al fine di fornire un sistema di notifiche riguardo ad alcune attività e recuperare dati riguardanti le partite precedentemente svolte.\newline

\noindent Inoltre, verrà fornita una chat con la quale i giocatori possano comunicare durante la partita.\newline

\section{Obiettivi}

\noindent Per la prioritizzazione dei nostri obiettivi utilizzeremo il metodo MoSCoW:
\begin{enumerate}
    \item \textit{MUST}:     
    \begin{itemize}
        \item Modellazione di tutti i sistemi tramite rappresentazioni di Ingegneria del Software
        \item Gestione delle lobby
        \item Gestione degli stati di gioco      
        \item Autentificazione
        \item Notifiche
        \item Test con utenti
        \item Resposiveness
        \item Accessibilità
    \end{itemize}
    \item \textit{SHOULD}: 
    \begin{itemize}
        \item Implementazione di un sistema di messaggistica
        \item Possibilità di consultare i report delle partite passate
        \item Feedback grafici (popup)
    \end{itemize}
    \item \textit{WOULD}:
        \begin{itemize}
            \item Animazioni
            \item Feedback audio
        \end{itemize}
    \item \textit{WON'T}:     
    \begin{itemize}
        \item Possibilità di scaricare in locale i report delle partite passate
    \end{itemize}
\end{enumerate}

\section{Personas}
Le personas sono veri e propri identikit di utenti ideali interessati al servizio offerto. Sono una rappresentazione dei tratti caratterizzanti di ciascun utente e di quelli che li accomunano.\newline

\noindent Per GuessR, sono state individuate 3 personas che identificano gli utenti target:

\subsection{Jonathan}
\begin{figure}[H]
    \centering
    \includegraphics[width=50mm]{img/personas/jonathan.jpg}
    \label{fig:personas_jonathan}
\end{figure}
\textit{Profilo personale:}
\begin{itemize}
    \item Età: 14 anni
    \item Sesso: maschile
    \item Status: vive con i suoi genitori e, durante le sere infrasettimanali, si collega con i suoi amici tramite Discord per giocare a vari videogiochi
\end{itemize}
\textit{Occupazione/Formazione:}
\begin{itemize}
    \item Frequenta l'ultimo anno di scuole medie, dove si concentrano anche la maggior parte dei suoi amici, molto simili a lui. Non dispone di un reddito.
\end{itemize}
\textit{Comportamento online:}
\begin{itemize}
    \item Tempo su internet: almeno 20 ore a settimana
    \item Utilizza internet per: messaggistica istantanea, social network, videogiochi, cercare notizie calcistiche
\end{itemize}
\textit{Interessi personali:}
\begin{itemize}
    \item Videogiochi: predilige i giochi di gruppo, ma occasionalmente gioca anche a dei single-player
    \item Calcio: segue tutte le notizie della sua squadra e frequenta un'associazione calcistica locale per dilettanti
\end{itemize}
\subsection{Jolyne}
\begin{figure}[H]
    \centering
    \includegraphics[width=50mm]{img/personas/jolyne.jpg}
    \label{fig:personas_jolyne}
\end{figure}
\textit{Profilo personale:}
\begin{itemize}
    \item Età: 22 anni
    \item Sesso: femminile
    \item Status: studentessa universitaria fuori sede, si riunisce spesso con i suoi amici per fare serata e, in quelle occasioni, necessita di giochi di gruppo semplici e divertenti
\end{itemize}
\textit{Occupazione/Formazione:}
\begin{itemize}
    \item Frequenta la triennale di psicologia, di tanto in tanto lavora in un ristorante a chiamata. Dispone di un reddito basso occasionale.
\end{itemize}
\textit{Comportamento online:}
\begin{itemize}
    \item Tempo su internet: 15 ore a settimana
    \item Utilizza internet per: messaggistica istantanea, social network, studio universitario
\end{itemize}
\textit{Interessi personali:}
\begin{itemize}
    \item Psiche umana: da sempre affascinata, sviluppa questa passione tramite il percorso di studi
    \item Arte: di tanto in tanto si cimenta in qualche illustrazione per svagarsi 
\end{itemize}
\subsection{Joseph}
\begin{figure}[H]
    \centering
    \includegraphics[width=50mm]{img/personas/joseph.jpg}
    \label{fig:personas_joseph}
\end{figure}
\textit{Profilo personale:}
\begin{itemize}
    \item Età: 27 anni
    \item Sesso: maschile
    \item Status: convive con la sua compagna in un appartamento, occasionalmente si connette tramite Skype con gli amici della sua città natale
\end{itemize}
\textit{Occupazione/Formazione:}
\begin{itemize}
    \item Dopo aver terminato le scuole superiori ha cominciato una carriera da impiegato d'ufficio e, una volta ricevuta una promozione, l'azienda l'ha trasferito fuori sede
\end{itemize}
\textit{Comportamento online:}
\begin{itemize}
    \item Tempo su internet: almeno 30 ore a settimana
    \item Utilizza internet per: messaggistica istantanea, social network, eseguire pratiche burocratiche a lavoro, riempire i momenti di vuoto lavorativi
\end{itemize}
\textit{Interessi personali:}
\begin{itemize}
    \item Film e serie TV: li utilizza la sera per rilassarsi
    \item Videogiochi: altra alternativa serale, videogiocatore dall'alba dei tempi
\end{itemize}
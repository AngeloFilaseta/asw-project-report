\chapter{Tecnologie}
Le tecnologie impiegate per il progetto in questione sono state tra le prime cose dibattute e definite, previa fase di modellazione dell'applicativo. Ogni tecnologia, framework e ambiente di sviluppo è stato scelto valutando la fattibilità, la coerenza e l'impiego ottimale di risorse oltre che all'ottimizzazione delle prestazioni. In questa prima fase ci siamo soffermati affinché potessimo garantire che ogni tecnologia utilizzata potesse permetterci di costruire un solido sistema distribuito, che pertanto avesse tutte le caratteristiche fondamentali di esso. In seguito andremo ad indicare e descrivere le principali tecnologie utilizzate suddividendole tra tecnologie condivise tra client e server, server side, client side e database.

\section{Tecnologie condivise tra client e server}
\subsection{Node Package Manager}
\noindent Node.js ~\cite{nodejs:online} è una runtime di JavaScript multipiattaforma per l'esecuzione di codice JavaScript, costruita sul motore JavaScript V8 di Google Chrome.\newline
Node.js dispone di una grande quantità di moduli scritti completamente in Javascript.
Essendo il progetto open source è inoltre possibile per gli sviluppatori aggiungere i propri moduli in modo da renderli disponibili pubblicamente ~\cite{nodejs:online}.\newline 
\noindent Il gestore di pacchetti predefinito per l'ambiente di runtime JavaScript Node.js si chiama Node Package Manager (npm).
npm può essere richiamato tramite linea di comando usando la seguente sintassi:
\begin{verbatim}
    npm <command> [args]
\end{verbatim}
Il comando base per ottenere un pacchetto è:
\begin{verbatim}
    npm install packet_name
\end{verbatim}
Tutte le dipendenze e i conflitti vengono gestiti automaticamente ~\cite{npmDo:online}.
\newline Grazie a Node è anche possibile per esempio creare un progetto React utilizzando il comando seguente:
\begin{verbatim}
    npx create-react-app project
\end{verbatim}
\textbf{npx} è uno strumento integrato in npm in grado di eseguire pacchetti, anche se non sono ancora installati nel sistema.\newline
Sarà poi possibile avviare il server di sviluppo utilizzando i seguenti comandi:
\begin{verbatim}
    cd project
    npm start
\end{verbatim}
L'interfaccia sarà poi visualizzabile all'indirizzo \texttt{http://localhost:3000} ~\cite{PrimiPas97:online}.

\begin{figure}[H]
    \caption{Logo di npm ~\cite{npm:online}}
    \centering
    \includegraphics[width=40mm]{img/logos/npm_logo.png}
    \label{fig:npm_logo}
\end{figure}

\subsection{Socket.io}
\noindent Socket.io è una libreria JavaScript che fornisce un servizio molto simile alle originali Web-Socket. Socket.io offre delle API JavaScript cross-browser che permettono la creazione di un canale di comunicazione full-duplex tra domini a bassa latenza tra il browser e il server web. \newline

\noindent Socket.io tenta di utilizzare prima di tutto le WebSocket native. In caso fallisca (in caso di incompatibilità coi sistemi o a causa di altri problemi)
ricorrerà all'utilizzo di polling HTTP. \newline

\noindent Socket.io è stato progettato per funzionare con tutti i browser moderni (97\% di compatibilità nel 2020) e in ambienti che non
supportano il protocollo WebSocket, ad esempio dietro proxy aziendali restrittivi. \newline

\noindent Nonostante il client di GuessR utilizzi a sua volta un ambiente javascript, Socket.io offre un supporto per la realizzazione di client in altri linguaggi quali Java, C++, Python, ecc...\newline
Grazie a tale supporto, sarà possibile implementare molteplici client che possano in modo agile interfacciarsi a tale mezzo di comunicazione.\newline


\noindent Oltre alle funzionalità offerte da una tradizionale WebSocket, Socket.io offre inoltre le seguenti features:
\begin{itemize}
    \item reliability (switch a polling HTTP nel caso la connessione WebSocket non possa essere stabilita)
    \item riconnessione automatica
    \item buffering dei pacchetti
    \item acknowledgments
    \item broadcast a tutti i client o ad un sottoinsieme di essi (Room)
    \item multiplexing
\end{itemize}

\noindent Le ultime due feature in particolare ben si sposano con GuessR:
\begin{itemize}
    \item Il concetto di Room trova una sua trasposizione uno a uno con il concetto di lobby di gioco, permettendo di scambiare i messaggi agilmente all'interno di tale lobby
    \item Tramite multiplexing è possibile definire molteplici namespace così da poter associare handler diversi e meglio separare le tipologie di messaggi scambiati tra client e server (ad esempio creando un namespace dedicato alla chat, un namespace dedicato all'inoltro di dati, ecc...)
\end{itemize}


\begin{figure}[H]
    \caption{Logo di Socket.io}
    \centering
    \includegraphics[width=50mm]{img/logos/socketIO_logo.jpg}
    \label{fig:sockIO_logo}
\end{figure}

\section{Server-side}
\label{sub:server-tech}


\subsection{Node.js}
Come runtime JavaScript guidato da eventi asincroni, Node.js è progettato per creare applicazioni di rete scalabili; molte connessioni, difatti, possono essere gestite contemporaneamente. Ad ogni connessione verrà invocata una callback, rendendo Node attivo solo al momento necessario.\newline

\noindent Nonostante in questo progetto Node sia stato utilizzato per sviluppare un server, è importante sottolineare che Node.js non è un web server, ma una piattaforma in grado di eseguire codice javascript \textit{lato server} e, tramite apposite librerie, sviluppare per l'appunto un web server.\newline

\noindent Node.js implementa un'architettura event-driven, facendo dunque affidamento su un event loop. Non esiste alcuna chiamata per avviare il ciclo: Node.js entra semplicemente nel ciclo degli eventi dopo aver eseguito lo script di input e, analogamente, esce dal ciclo di eventi quando non ci sono più callback da eseguire. Questo comportamento è simile a JavaScript in browser: il ciclo degli eventi è nascosto all'utente.\newline

\noindent Per natura dell'event loop, Node è single-threaded, ma è possibile, su neccessità, effettuare delle fork per sfruttare al meglio i core offerti dalla macchina creando nuovi thread.
\begin{figure}[H]
    \caption{Logo di Node.js ~\cite{Nodejs66:online}}
    \centering
    \includegraphics[width=70mm]{img/logos/nodejs_logo.png}
    \label{fig:node_js_logo}
\end{figure}

\subsection{Mongoose}
Mongoose è una libreria per Node.js che permette di creare degli Schema per rappresentare i dati da archiviare nel sistema.\newline
Ogni Schema è associato ad una collezione nel Database di MongoDB.\newline
Mongoose viene utilizzato per la creazione del proprio model, è infatti possibile creare delle istanze dallo Schema attraverso delle Factory ed utilizzarli come dei semplici oggetti Javascript.\newline
Oltre ad offrire metodi aggiuntivi già pronti per salvare i dati dentro il Database è possibile creare funzioni che sono oggetti appartenenti ad un relativo schema possono richiamare, rendendo la modellazione molto object-oriented.\newline
La creazione di uno schema di esempio per archiviare gattini su Mongoose avviene nel seguente modo:
\begin{minted}{js}
const kittySchema = new mongoose.Schema({
  name: String
});
\end{minted}
Dopo aver creato uno Schema è necessario creare una costante che rappresenta il model e collegarlo allo Schema appena creato:
\begin{minted}{js}
const Kitten = mongoose.model('Kitten', kittySchema);
\end{minted}
Ora è possibile creare dei nuovi gattini utilizzando una sintassi object oriented:
\begin{minted}{js}
let jojo = new Kitten({ name: 'JoJo' });

console.log(jojo.name); // 'JoJo'
\end{minted}
È anche possibile creare funzioni da associare al model appena creato:
\begin{minted}{js}
kittySchema.methods.meow = function () {
  const greeting = this.name
    ? "Meow name is " + this.name
    : "I don't have a name";
  console.log(greeting);

jojo.meow(); // 'Meow name is JoJo'
\end{minted}
Per salvare il gattino nella collezione basterà richiamare il metodo \textit{save()}. Si consiglia di utilizzare i metodi \textit{then()} e \textit{catch()} per avere controllo sul risultato dell'operazione:
\begin{minted}{js}
jojo.save() //save jojo in the collection
    .then() //do something when operation is completed
    .catch() //do something if an error occurred

\end{minted}

\begin{figure}[H]
    \caption{Logo di Mongoose ~\cite{Mongoose21:online}}
    \centering
    \includegraphics[width=70mm]{img/logos/mongoose_logo.png}
    \label{fig:mongoose_logo}
\end{figure}



\section{Client-side}
Per il lato client è stato scelto uno stack tecnologico che ci permettesse il più possibile il riutilizzo di codice e l'automazione.\newline
È stato dunque scelto l'utilizzo di javascript con l'appoggio di node.js grazie ai quali è stato possibile realizzare agilmente sia la versione web del client che quella mobile nativa.\newline
\label{sub:client-tech}

\subsection{React.js}
React è un framework open-source che permette di implementare applicazioni web seguendo i principi della
programmazione ad oggetti. In modo particolare risulta essenziale comprendere tre
concetti chiave:
\begin{itemize}
    \item \textbf{JSX}\\
    JSX ~\cite{introduzione_jsx_react} è un'estensione della sintassi di JavaScript.\newline
    Permette di unire gli aspetti di html come linguaggio di template agli aspetti di JavaScript come linguaggio di scripting in una forma che ne aumenta semplicità e leggibilità.\newline
    Gli elementi JSX vengono utilizzati nelle definizioni delle funzioni di rendering (che verranno trattate a breve) semplificando la costruzione della user interface.\newline
    Attraverso JSX possiamo richiamare un componente React tramite con un meccanismo analogo ai tag in html.
    \item  \textbf{Componenti}\\
    Attraverso un componente ~\cite{componente_react} andiamo a definire quella che risulta essere a tutti gli effetti una classe.\newline
    Il nostro componente, definito da uno o più costruttori, contiene uno stato; questo stato verrà mantenuto, ed eventualmente aggiornato, durante tutto il ciclo di vita (lifecycle) del componente.\newline
    È possibile ricevere dati ed istruzioni da altri componenti attraverso le props.\newline
    \item \textbf{Stato}\\
    Lo stato ~\cite{state_e_lifecycle_react} un insieme di proprietà di un componente.\newline
    Queste proprietà possono variare a seguito dell'interazione con altri componenti o come azione del componente stesso nel caso esso esegua delle azioni a cadenza temporale.\newline
    In React inoltre lo stato risulta fondamentale ai fini di ottenere un rendering a schermo performante; ogni componente React infatti deve obbligatoriamente definire una funzione \emph{render()}.\newline
    Attraverso di essa verrà ritornato il contenuto da renderizzare a schermo.\newline 
    React cambierà il contenuto a schermo (consumando quindi risorse) solamente quando vi saranno delle modifiche nel contenuto ritornato dalla funzione render(); utilizzando dunque le proprietà che definiscono lo stato del componente all'interno della funzione render() del componente stesso riusciremo a ridurre al minimo indispensabile il numero di volte in cui l'applicazione verrà renderizzata, riflettendo i cambiamenti di stato del componente.    
\end{itemize}

\noindent React introduce anche i componenti funzione, che di fatto svolgono la stessa funzione dei componenti precedentemente spiegati (componenti classe), ma con una sintassi più concisa.

\begin{figure}[H]
    \caption{Logo di React.js ~\cite{React:online}}
    \centering
    \includegraphics[width=40mm]{img/logos/react_logo.png}
    \label{fig:react_logo}
\end{figure}
\subsection{Redux}
Redux ~\cite{caratteristiche_redux} è un \emph{contenitore di stato} per applicazioni JavaScript.\newline
Gode di quattro caratteristiche fondamentali per progetti portata medio-grande:
\begin{itemize}
  \item \textbf{Deterministico}\\
  Un aspetto fondamentale nelle applicazioni web di ogni genere è il determinismo.\newline
  L'oneroso compito di far collaborare tutti i componenti al fine di ottenere un comportamento predicibile viene largamente semplificato dal framework Redux.
  \item \textbf{Centralizzato}\\
  Avere stato e logica centralizzati permette di ottenere rapidamente delle feature di fondamentale importanza, come funzioni di "annulla" e "ripeti" che ci permettono di muoverci agilmente tra lo storico degli stati.\newline
  Inoltre un approccio centralizzato garantisce un consistenza maggiore dei dati, rendendo possibile modificarli solamente attraverso specifiche funzioni (azioni) create durante la definizione della struttura dati.
  \item \textbf{Debug oriented}\\
  Attraverso semplici plugin browser come Redux DevTools risulta immediato il debug dell'applicazione.\newline
  Difatti attraverso questi tool otteniamo una visione completa dello stato dei dati e della sua evoluzione nel tempo, riuscendo ad identificare con precisione quando e soprattutto perchè lo stato abbia subito delle modifiche.
  \item \textbf{Flessibile}\\
  Redux funziona con ogni layer di User Interface e, essendo ormai uno strumento consolidato, dispone di un solido supporto e un vasto parco di plugin e pacchetti aggiuntivi per ogni esigenza.
\end{itemize}
Per comprendere come queste caratteristiche si concretizzino risulta fondamentale comprendere tre concetti chiave nella struttura di Redux:
\begin{itemize}
  \item \textbf{Store}\\
  Lo store ~\cite{store_redux} è un oggetto che contiene l'intera struttura ad albero dello stato.\newline
  Fornisce metodi per la lettura dello stato corrente.
  \item \textbf{Reducer}\\
  I reducer ~\cite{reducer_redux} definiscono la struttura dello store.\newline
  Vi possono essere più reducer all'interno di una stessa applicazione, al fine di meglio suddividere lo stato.
  \item \textbf{Azioni}\\
  Le azioni ~\cite{azioni_redux} permettono di modificare il contenuto dello store.\newline
  Essendo queste l'unico modo per alterare lo stato attuale, qualsiasi componente che voglia agire sullo stato deve passare per le azioni definite.
\end{itemize}
Ma \emph{perchè} utilizzare un contenitore di stato se ogni componente React possiede già uno stato?
React, pur essendo uno strumento molto potente, ci pone davanti a delle limitazioni:
\begin{itemize}
  \item
  Innanzitutto non è raro che più componenti debbano fare riferimento allo stesso dato; la presenza di uno stato comune evita di dover implementare meccanismi ad hoc di sincronia tra gli stati dei due componenti.\newline
  Lo stato di ogni componente, attraverso tool ready to use, sarà dunque sincronizzato con la struttura principale.
  \item
  In secondo luogo React, data la sua natura orientata al reactive programming, incoraggia la propagazione dell'informazione solo in una direzione: da un componente padre verso un componente figlio (utilizzato dunque dal padre).\newline
  La presenza delle azioni Redux risolvono questo problema in quanto ogni cambiamento apportato allo stato Redux si rifletterà su tutti i componenti React che utilizzano quel particolare dato.
\end{itemize}
\begin{figure}[H]
    \caption{Logo di Redux ~\cite{redux:online}}
    \centering
    \includegraphics[width=70mm]{img/logos/redux_logo.png}
    \label{fig:redux_logo}
\end{figure}

\subsection{Bootstrap}
\noindent Bootstrap è stato creato nel 2010, tramite Twitter, da @mdo and @fat.\newline
È rapidamente diventato il framework più popolare per il front-end a livello mondiale.
Con le prime release della versione 2 vennero introdotti i primi fogli di stile (opzionali) per aggiungere funzionalità riguardanti il design responsive.\newline
Con l'uscita della versione 3 l'intera libreria è stata ripensata e riscritta per offrire supporto al design responsive con un approcio mobile-first.
La versione 4 di Bootstrap infine (quella attuale) ha subito due principali cambiamenti nella sua architettura:
\begin{itemize}
    \item una migrazione a Sass, un'estensione del linguaggio CSS che permette l'uso di variabili e di funzioni. L'estensione dei file Sass è .scss;
    \item l'uso di CSS flexbox, ovvero contenitori per elementi.
\end{itemize}

\noindent Bootstrap adotta un approcio mobile-first.
L'assunzione di base è che la maggior parte dei visitatori utilizzeranno uno smartphone a causa del sempre più crescente utilizzo di questo tipo di dispositivi.\newline
L'utilizzo continuo di media query è alla base di ogni layout creato utilizzando Bootstrap ~\cite{bootstrap}. \newline Moltissime classi offerte utilizzano i cosiddetti breakpoints. Si tratta di meccanismi che permettono di nascondere, mostrare o modificare il comportamento degli elementi a cui vengono applicate in base alla risoluzione dello schermo del visitatore.
Bootstrap mette a disposizione i seguenti breakpoints:
\begin{minted}{css}
// Small devices (landscape phones, 576px and up) KEYWORD:sm
@media (min-width: 576px) { ... }

// Medium devices (tablets, 768px and up) KEYWORD:md
@media (min-width: 768px) { ... }

// Large devices (desktops, 992px and up) KEYWORD:lg
@media (min-width: 992px) { ... }

// Extra large devices (large desktops, 1200px and up) KEYWORD:xl
@media (min-width: 1200px) { ... } 
\end{minted}
\noindent Segue un rapido esempio per l'uso dei breakpoints. Per la gestione dei margini e del padding Bootstrap mette a disposizione le classi di tipo \{property\}\{sides\}-\{breakpoint\}-\{size\}.
\begin{itemize}
    \item \textbf{\{property\}}: \textbf{m} se si vuole aggiungere un margine, \textbf{p} se si vuole aggiungere del padding;
    \item \textbf{{\{sides\}}}: \textbf{t} per top, \textbf{b} per bottom, \textbf{r} per right, \textbf{l} per left, \textbf{x} per right e left, \textbf{y} per top e bottom;
    \item \textbf{\{breakpoints\}}: se omesso, la classe funziona su qualsiasi tipo di schermo, anche i più piccoli, altrimenti funziona solo sugli schermi più grandi anche solo di un pixel del breakpoint specificato;
    \item \textbf{\{size\}}: Bootstrap non utilizza valori statici per specificare di quanto deve essere il padding o il margin. Il valore che può assumere size va da 0 a 5, e sono valori che indicano quando deve essere grande il moliplicatore che gestisce lo spazio. 0 significa nessun padding/margin.
    
\end{itemize}
\noindent Se si desidera applicare un piccolo margine a destra ad un determinato elemento solo se il visitatore lo visualizza su uno schermo con larghezza maggiore di 992px, le classi che l'elemento dovrà assumere saranno \textbf{mr-0 mr-lg-2}.\newline
\textbf{mr-0} significa nessun margine a destra su qualsiasi tipo di schermo. \textbf{mr-lg-2} invece vuol dire margine a destra con proporzione 2 solo su schermi più grandi di 992px. Combinando insieme le due classi otteniamo un margine solo su schermi con larghezza maggiore di 992px ~\cite{bootstrapDoc}.
\begin{figure}[H]
    \centering
    \includegraphics[width=0.2\linewidth]{img/logos/bootstrap_logo.png}
    \caption{Logo di Bootstrap ~\cite{Bootstra56:online}.}
    \label{bootstrapLogo}
\end{figure}

\section{Database-side}
\subsection{mongoDB}

MongoDB è un DBMS NoSQL, cioè non utilizza un meccanismo di persistenza relazionale come un tradizionale SQL.\newline
Il modello NoSQL non è unico e può dunque utilizzare varie strutture dati per sostiturire le tabelle con campi uniformi usate in SQL.\newline
In particolare mongoDB utilizza un modello orientato al documento, dove le informazioni sono memorizzate in una struttura gerarchica ad albero ed un qualsiasi numero di campi con qualsiasi lunghezza può essere aggiunto. I campi a loro volta possono anche contenere pezzi multipli di dati.\newline

\noindent I DBMS orientati al documento offrono alcuni vantaggi, specialmente in ambito web, rispetto ai tradizionali RDBMS:
\begin{itemize}
    \item Maggiore flessibilità dei dati, utile per avere meno rigidità in fase di sviluppo o, in generale, per scenari in cui i dati memorizzati non sono sempre uniformi
    \item Facilità nella trasposizione in strutture dati javascript in quanto i JSON (i documenti utilizzati internamente in mongoDB e, ormai, standard de facto) trovano una corrispondenza uno a uno con esse.
\end{itemize}

\noindent D'altro canto è pur vero che una struttura meno rigida aumenta il rischio di duplicazione di dati ed inconsistenze; è dunque richiesta al progettista una maggiore cautela nella manipolazione di dati.
\begin{figure}[H]
    \centering
    \includegraphics[width=0.5\linewidth]{img/logos/mongo_logo.png}
    \caption{Logo di mongoDB ~\cite{Ildataba34:online}.}
    \label{mongoLogo}
\end{figure}

\newpage
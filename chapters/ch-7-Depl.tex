\chapter{Deployment}
In questo capitolo verrà illustrato come poter effettuare il deploy dell'applicativo.\newline

\noindent Oltre alla versione in locale, utilizzata in fase di sviluppo e testing, è stato possibile di rendere disponibile una versione online funzionante, così da testarne in maniera più accurata alcuni aspetti e ottenere feedback da utenti reali facendoli interagire con il sistema. \newline
\noindent Per tale scopo è stato utilizzato il servizio gratuito di Amazon AWS per ottenere una macchina virtuale sul cui poter effettuare il deploy.\newline

\noindent Verranno esposte due modalità per effettuare il deployment: una tradizionale ed una con il supporto di Docker.

\section{Docker-free deployment}
\subsection{Prerequisiti, installazione di Software}
\noindent Per poter effettuare il deployment risulta necessario installare i seguenti software permettere al sistema un corretto funzionamento:
\begin{itemize}
    \item \textbf{git}: necessario per reperire il repository del progetto. È possibile reperire i sorgenti in qualsiasi altro modo, ma utilizzare git permette ovviamente più flessibilità.
    \item \textbf{npm}: necessario per installare le dipendenze necessarie e per il corretto funzionamento del componente client.
    \item \textbf{node}: necessario per il corretto funzionamento del componente server.
    \item \textbf{mongodb}: necessario per avere a disposizione il database.
\end{itemize}
\subsection{Deployment Server}
Il deployment del server è molto più semplice della controparte client. È sufficiente collocarsi nella cartella del server, installare le dipendenze utilizzando:
\begin{minted}{bash}
   npm install
\end{minted}
ed avviare il server utilizzando il comando 
\begin{minted}{bash}
   node app.js
\end{minted}

\begin{warn}[ATTENTZIONE:]
nel caso si scelga di effettuare il deployment docker-free sarà necessario adeguare la stringa di connessione al db, reperibile presso asw-project/server/conf/conf.js
\end{warn}

\subsection{Deployment Client}

Per il deployment del client sarà necessario spostarsi all'interno della cartella "client" e lanciare il comando

\begin{minted}{bash}
    npm install
\end{minted}

che provvederà all'import di tutte le librerie npm utilizzate dal client.

È poi necessario eseguire il comando:
\begin{minted}{bash}
    npm run build
\end{minted}
che si occuperà di creare i file di build all'interno della cartella \textit{build}~\cite{ProductionReact:online}.\newline
Infine basterà eseguire il comando:
\begin{minted}{bash}
    serve -s build
\end{minted}

\section{Dockerized deployment}

\subsection{Docker}
Docker è un sistema open-source tramite il quale è possibile automatizzare il processo di deployment di applicazioni all'interno di contenitori software.\newline
Docker implementa API di alto livello per gestire \textit{container che eseguono processi in ambienti isolati}.\newline

\noindent Utilizzando i container dunque le risorse possono essere isolate, i servizi limitati e i processi avviati in modo da avere una prospettiva completamente privata del sistema operativo, col loro proprio identificativo, file system e interfaccia di rete. Più container condividono lo stesso kernel, ma ciascuno di essi può essere costretto a utilizzare una certa quantità di risorse, come la CPU, la memoria e l'I/O.\newline

\noindent L'utilizzo di Docker per creare e gestire i container può semplificare la creazione di sistemi distribuiti, permettendo a diverse applicazioni o processi di lavorare in modo autonomo sulla stessa macchina fisica o su diverse macchine virtuali. Ciò consente di effettuare il deployment di nuovi nodi solo quando necessario.\newline

\noindent Al fine di creare un container Docker sarà necessario specificare un \textbf{Dockerfile} per ogni servizio erogato. 
Similmente ai gitignore per il versioning git, è possibile definire dei \textbf{dockerignore} per segnalare a docker quali file ignorare durante la copia dei file all'interno del container.

\begin{figure}[H]
    \caption{Logo di Docker}
    \centering
    \includegraphics[width= 100mm]{img/logos/docker_logo.png}
    \label{fig:docker_logo}
\end{figure}

\subsection{Docker compose}
Il deployment di più sottosistemi può essere effettuato tramite lo strumento Compose di Docker; tramite essi verrà eseguita un'applicazione Docker multi-container.\newline

\noindent Sarà necessario creare un file \textbf{docker-compose.yml} dove verrà fatto riferimento ai Dockerfile di ogni singolo servizio al fine di creare un macro-container che contenga al suo interno tutti i container dei servizi.\newline

\noindent In quanto di default ogni container risulta isolato dagli altri, sarà necessario specificare l'esistenza di una sotto-rete interna all'esecuzione del macro-container del Docker compose. Gli indirizzi da specificare all'interno degli applicativi dovranno tenere conto dei namespace della rete così definita in docker-compose.yml.\newline

\noindent Per fare in modo che i servizi siano raggiungibili dall'esterno del container, sarà fondamentale inoltre mappare le porte interne al container con le porte della macchina sulla quale il container è in esecuzione.

\subsection{Deployment}
Una volta installato Docker sul sistema, sarà necessario semplicemente posizionarsi nella cartella asw-project ed eseguire il comando 
\begin{minted}{bash}
    docker-compose up --build
\end{minted}
Per terminare l'esecuzione sarà semplicemente necessario usare la combinazione di tasti CTRL+C e successivamente lanciare il comando
\begin{minted}{bash}
    docker-compose down
\end{minted}
Per far sì che lo script sia lanciato in modalità detached basterà lanciare il comando
\begin{minted}{bash}
    docker-compose up --build -d
\end{minted}
così che l'esecuzione del processo sia indipendente dalla bash presso la quale è stato lanciato il comando.\newline
\noindent In questo caso per terminare l'esecuzione sarà necessario solamente il comando docker-compose down.
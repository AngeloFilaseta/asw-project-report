\chapter{Conclusioni}
\section{Note finali}
L'elaborato finale sicuramente soddisfa tutti gli obiettivi imposti nella fase di definizione di progetto, in alcuni casi supera le aspettative, il tutto ponendo la giusta importanza alle varie componenti di gioco e rispettando la scala di priorità definita in una prima fase. \newline

\noindent Al termine di questo progetto possiamo affermare che si è realizzata un'architettura efficiente e a nostro parere ben strutturata, con una particolare attenzione riguardo alla user experience, all'accessibilità e in generale ad un'attività ed un'esperienza di gioco inclusive e implementate a regola d'arte. \newline

\noindent Ogni servizio svolge il suo compito correttamente garantendo il risultato atteso e si coordina con gli altri senza problemi. \newline

\noindent Al di fuori dell'elaborato, uno dei più grandi goal che ci siamo imposti inizialmente era far si che questo progetto ci permettesse di rimanere in contatto virtuale con amici, che non potremmo vedere a causa delle forti limitazioni alla mobilità dovute alla pandemia in atto. Anche in tal fronte ci possiamo ritenere soddisfatti e confermare di aver ritrovato lo stesso entusiasmo che abbiamo avuto noi nel crearlo durante il gioco.

\subsection{Sviluppi futuri}
Degli sviluppi abbastanza estensivi potrebbero vedere un tipo di interazione utente-utente, come scambio di amicizie o creazione di gruppi per velocizzare l'invito in lobby, oppure la personalizzazione di più parametri di gioco da parte dell'admin o di ogni utente, ad esempio la possibilità di scegliere il tempo da lasciare ad ogni utente per produrre una frase o un disegno.\newline

\noindent Per quanto riguarda i report uno degli sviluppi più immediati potrebbe essere l'aggiunta per ogni utente di scegliere se salvare i report o meno ed eventualmente dare la possibilità di eliminare alcuni dei report salvati, qualora non siano particolarmente d'interesse per l'utente. \newline

\noindent Ulteriori sviluppi sotto altri fronti che contribuirebbero a migliorare ulteriormente UI/UX e personalizzazione delle schermate utente potrebbero vedere l'integrazione di meccanismi di gamification (come avatar per gli utenti, badges e/o reward system), oppure l'immissione di nuove dinamiche come l'introduzione di classifiche basate sul voto degli utenti, che potrebbero giudicare le frasi e i disegni migliori.

\newpage